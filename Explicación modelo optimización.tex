\documentclass[letterpaper]{article}
\usepackage[spanish]{babel}
\selectlanguage{spanish}
\usepackage[utf8]{inputenc}

\usepackage{lipsum}
\usepackage{amsmath,amssymb,amsfonts,amsbsy}
\usepackage{array}
\usepackage{graphicx}
\usepackage{subfigure}
\usepackage{float}
\usepackage{hyperref}

\graphicspath{ {./figures/} }

%\usepackage[pass]{geometry}
\usepackage[left=1.25in,right=1.25in,top=1.0in,bottom=1.0in]{geometry}
\usepackage{listings}

% Custom colors
\usepackage{color}
\definecolor{deepblue}{rgb}{0,0,0.65}
\definecolor{deepred}{rgb}{0.7,0,0}
\definecolor{deepgreen}{rgb}{0,0.6,0}

\newcommand{\mytitle}{Tarea 1}
\newcommand{\myauthor}{Mariana Ortega}
\newcommand{\mydate}{\today}

\begin{document}

\hfill
\vspace{0pt}
\hfill
\vspace*{6cm}
\begin{center}{}
\vspace*{2mm}
\vspace*{4mm}
\hrule\vspace*{1pt}\hrule
\vspace*{4mm}
{\LARGE\bf Generador de horarios para colegios\\
\vspace*{4mm}
\vspace*{1mm}
\end{center}

\vspace*{50mm}

\newpage

\tableofcontents

\newpage

\section{Explicación}
El proyecto tiene como propósito modelar el problema de generación de horarios como un modelo de optimización, teniendo en cuenta las restricciones que puede tener un problema de este tipo y las que podría querer considerar un colegio al momento de hacer los horarios. Luego, se quiere generar un programa que permita a la persona que quiere generar su horario, personalizar este, imponiendo restricciones y viendo si estas se pueden cumplir. Para la interfaz se propondrán algunas ideas.

\section{Modelamiento del problema}

\subsection{Datos}
\begin{itemize}
    \item $dias$ \textbf{(D)}: Lista que muestra los días en los cuales se quieren planificar clases en un colegio. Generalmente será de lunes a viernes.
    \item $modulos$ \textbf{(M)}: Lista que contiene todos los módulos de clases con los que cuenta el colegio.
    \item $combinaciones$ \textbf{(CMB)}: Lista que contiene listas de la forma $[i,j]$ que señala que entre el ramo $i$ y el ramo $j$ no hay ninguna pausa (recreo o almuerzo).
    \item $horarios\,[curso]\,[día]$ \textbf{(H)}: Para cada día de cada curso muestra los módulos de clases que tiene cada curso en cada día de la semana.
    \item $limitaciones\_profesores\,[profesor] \textbf{(LP)}$: Para cada profesor entrega el máximo de módulos de clases que puede hacer en una semana como también el máximo de módulos que puede hacer en un día.
    \item $carga\_profesores\,[profesor]$ \textbf{(CP)}: Para cada profesor entrega, en una lista, el/los curso/s a los que se le hace una clase y el nombre del ramo.
    \item $profesores\_por\_curso\,[curso]$ \textbf{(PpC)}:Para cada curso se tiene una lista con la lista de los profesores que hacen un ramo y el nombre del ramo.
    \item $profesores\_a\_definir\_por\_curso\,[curso]$ \textbf{(PadpC)}:Para cada curso se tiene una lista con la lista de los profesores que podrían hacer el ramo y el nombre del ramo. Solo se elegirá uno de los profesores.
    \item $ramos\_por\_curso \, [curso] \, [ramo]$ \textbf{(RpC)}: Para cada ramo de cada curso indica la cantidad de módulos semanales, la cantidad máxima de módulos diarios y el tipo de módulos (seguidos, disjuntos o sin preferencia).
    \item $asignaciones$ \textbf{(A)}: Lista que contiene todas las tuplas de la forma curso, profesor, clase.
    \item $asignaciones\_con\_restricciones\_de\_dias$ \textbf{(AcR)}: Lista que contiene todas las tuplas de la forma curso, profesor, clase que se quiere que tengan módulos consecutivos
    \item $profesores\_con\_dias\_limitados$ \textbf{(PDL)}: Lista de profesores que solo harán una cantidad de días clases.
    \item $tope\_dias\,[profesor]$ \textbf{(TD)}: Cantidad máxima de días que cada profesor hace clases. Esto es solo para los profesores que tienen tope.
\end{itemize}

\subsection{Variables}
\begin{itemize}
    \item $x_{cprdj}$: Toma el valor $1$ si el curso $c$ tiene el ramo $r$ con el profesor $p$ el día $d$ en el módulo $j$ y $0$ en caso contrario.
    \item $s_{cprd}$: Toma el valor $1$ si el curso $c$ tiene el ramo $r$ con el profesor $p$ el día $d$ y $0$ en caso contrario.
    \item $y_{pd}$: Toma el valor $1$ si el profesor $p$ hace clases el día $d$ y $0$ en caso contrario.
    \item $w_{cpr}$: Toma el valor $1$ si el profesor $p$ hace el ramo $r$ al curso $c$.
\end{itemize}

\subsection{Restricciones}

\begin{enumerate}
    \item Tiene que haber como máximo una clase cada módulo de cada día que forma parte del horario de cada curso:

    $$\sum_{\forall ps,r \in PpC[c]} x_{c,ps[0],r,d,j} + \sum_{\forall ps,r \in PadpC[c]} \sum_{\forall p \in ps} x_{c,p,r,d,j} \leq 1 \; \forall c \in H, \forall d \in H[c], \forall j \in H[c][d]$$

    \item Tiene que haber como mínimo un profesor para cada módulo de cada día que forma parte del horario de cada curso:

    $$\sum_{\forall ps,r \in PpC[c]} \sum_{p \in ps} x_{c,p,r,d,j} + \sum_{\forall ps,r \in PadpC[c]} \sum_{\forall p \in ps} x_{c,p,r,d,j} \geq 1 \;  \forall c \in H, \forall d \in H[c], \forall j \in H[c][d]$$

    \item Si más de un profesor hace una clase, entonces estos la hacen simultáneamente:
    
    $$x_{c,p_1,r,d,j} = x_{c,p_2,r,d,j} \; \forall c \in PpC, \forall ps,r \in PpC[c] : len(ps) > 1, \forall p_1,p_2 \in ps, \forall d \in D, \forall j \in M$$

    \item El horario incluye la cantidad de horas exacta que se quiere por ramo:
    
    $$\sum_{\forall d \in D} \sum_{\forall j \in M} x_{c,p,r,d,j} = RpC[c][r]['modulos\_semanales'] \; \forall c \in PpC, \forall ps,r \in PpC[c], \forall p \in ps$$

    \item El horario incluye la cantidad de horas exacta que se quiere por ramo (caso profesores no asignados):
    
    $$\sum_{\forall d \in D} \sum_{\forall j \in M} \sum_{\forall p \in ps} x_{c,p,r,d,j} = RpC[c][r]['modulos\_semanales'] \; \forall c \in PpC, \forall ps,r \in PadpC[c]$$

    \item No se tiene clases en los módulos que no son parte del horario:

    $$\sum_{ps,r \in PpC[c]} \sum_{p \in ps} x_{c,p,r,d,j} = 0 \; \forall c \in H, \forall d \in H[c], \forall j \in M \setminus H[c][d]$$

    \item Los profesores solo pueden hacer una clase al mismo tiempo (sin contar las que se hacen a más de un curso al mismo tiempo):

    $$\sum_{cs,r \in CP[p]} x_{cs[0],p,r,d,j} \leq 1 \; \forall p \in CP, \forall d \in D, \forall j \in M$$

    \item Vinculación clases que profesores hacen a un más de un curso simultáneamente.
    
    $$x_{c_1,p,r,d,j} = x_{c_2,p,r,d,j} \; \forall p \in CP, \forall cs,r \in CP[p], \forall c_1,c_2 \in cs, \forall d \in D, \forall j \in M$$

    \item Se tiene como máximo la cantidad de módulos diarios definidos:

    $$\sum_{\forall j \in M} x_{c,ps[0],r,d,j} \leq RpC[c][r]['maximo\_modulos\_diario'] \; \forall c \in PpC, \forall ps,r \in PpC[c], \forall d \in D$$

    \item Se tiene como máximo la cantidad de módulos diarios definidos (caso profesores no asignados):

    $$\sum_{\forall j \in M} \sum_{\forall p \in ps} x_{c,p,r,d,j} \leq RpC[c][r]['maximo\_modulos\_diario'] \; \forall c \in PpC, \forall ps,r \in PpC[c], \forall d \in D$$

    \item La misma clase solo puede ocurrir el mismo día en módulos consecutivos si se eligió esa opción:

    $$x_{c,p,r,d,j} + x_{c,p,r,d,z} \leq 1 \; \forall c,p,r \in A : RpC[c][r]['tipo\_modulos'] = ``seguidos", \forall d \in D, \forall j,z \in \left((H \times H) \setminus CMB \right)$$

    \item La misma clase solo puede ocurrir el mismo día en módulos disjuntos si se eligió esa opción:

    $$x_{c,p,r,d,j} + x_{c,p,r,d,z} \leq 1 \; \forall c,p,r \in A : RpC[c][r]['tipo\_modulos'] = ``disjuntos", \forall d \in D, \forall j,z \in CMB)$$

    \item Introducción de la variable $s$ para preferir módulos dobles. Esta variable solo existe para los ramos de cada curso donde se quiere módulos seguidos:

    $$M * s_{c,p,r,d} \geq \sum_{j \in M} x_{c,p,r,d,j} \; \forall c,p,r \in AcR, \forall d \in D$$

    \item Uso de la variable $s$ para forzar módulos dobles. Uso opcional:

    $$\sum_{d \in D} s_{c,p,r,d} \leq \left\lceil \frac{RpC[c][r]['modulos\_semanales']}{2} \right\rceil \; \forall c,p,r \in AcR$$

    \item Los profesores no pueden hacer más módulos semanales que el máximo definido.
    
    $$\sum_{\forall cs,r \in CP[p]} \sum_{\forall j \in M} \leq LP[p]['maximo\_modulos\_diario'] \forall p \in CP \forall d \in D$$

    \item Los profesores no pueden hacer más módulos diarios que el máximo definido.
    
    $$\sum_{\forall cs,r \in CP[p]} \sum_{\forall j \in M} \leq LP[p]['maximo\_modulos\_diario'] \forall p \in CP \forall d \in D$$

    \item Ningún profesor hace más de 4 módulos seguidos:

    $$\sum_{cs,r \in CP[p]} x_{cs[0],p,r,d,j_1} + x_{cs[0],p,r,d,j_2} + x_{cs[0],p,r,d,j_3} + x_{cs[0],p,r,d,j_4} + x_{cs[0],p,r,d,j_5} \leq 4$$
    $$\forall p \in CP, \forall d \in D, \forall j_1,j_2,j_3,j_4,j_5 \in [(i,i+1,i+2,i+3,i+4) \forall i \in M : i+4 \in M]$$
    
    \item Introducción de la variable $y$ para poner un tope a la cantidad de días que los profesores con días limitados pueden hacer clases:

    $$M * y_{p,d} \geq \sum_{cs,r \in CP[p]} \sum_{\forall j \in M} x_{cs[0],p,r,d,j} \; \forall d \in D, \forall p \in PDL$$

    \item Uso de la variable $y$ para limitar la cantidad de días que un profesor hace clases:

    $$\sum_{\for d \in D} y_{p,d} \leq TD[p] \; \forall p \in PDL$$

    \item De los profesores que no se han definido solo se elige uno:

    $$\sum_{\forall ps,r \in PadpC[c]} \sum_{\forall p \in ps} x_{c,p,r,d,j} \leq 1 \; \forall c \in H, \forall d \in H[c] \forall j \in H[c][d]$$

    \item Implementación de la variable w:

    $$M * w_{c,p,r} \geq \sum_{\forall d \in H[c] \forall j \in H[c][d]} \; \forall c \in PadpC, \forall ps,r \in PadpC[c], \forall p \in ps$$

    \item Máximo un profesor para los ramos de cada curso que no tienen profesor asignado:

    $$\sum_{\forall ps,r \in PadpC[c]} \sum_{\forall p \in ps} \leq 1, \; \forall c \in PadpC$$
\end{enumerate}

\end{document}

